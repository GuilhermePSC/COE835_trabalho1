%---------------------------------------------------------------------
\section{Prova da limita��o de $\dot{\varepsilon}/m$}

Utilizando as equa��es do sistema, a din�mica do erro � expressa por:
%
\begin{flalign}
\dot{e}_0 &= \dot{y}_p - \dot{y}_m \nonumber \\
&= -a_m \, e_0 + \tilde{\theta} \, y_p \,,
\end{flalign}
%
ou ainda:
%
\begin{flalign}
e_0 &= \mathcal{L}^{-1}\left\{\frac{1}{s+a_m}\right\} \circledast ( \tilde{\theta}\,y_p ) \nonumber \\ 
&= \mathcal{L}^{-1}\left\{\frac{1}{s+a_m}\right\} \circledast (\theta\,y_p) - \theta^* \, \zeta \,,
\end{flalign}
%
onde $\tilde{\theta} = \theta - \theta^*$, $\theta^*$ � o ganho de realimenta��o ideal, o operador $\mathcal{L}^{-1}\{*\}$ denota a transformada de Laplace inversa e $\circledast$ denota a opera��o de \textit{convolu��o}.
%
Assim, o erro de estimativa pode ser escrito como:
%
\begin{flalign}
\varepsilon &= e_0 - \hat{e} \nonumber \\ 
&=\mathcal{L}^{-1}\left\{\frac{1}{s+a_m}\right\} \circledast (\theta \, y_p) - \theta^* \, \zeta - \mathcal{L}^{-1}\left\{\frac{1}{s+a_m}\right\} \circledast (\theta \, y_p) + \theta \, \zeta \nonumber \nonumber \\
&= \tilde{\theta} \, \zeta \,.
\label{eq:erro_est}
\end{flalign}

A seguinte candidata � fun��o de Lyapunov � proposta:
%
\begin{equation}
2V(\tilde{\theta}) = \gamma^{-1} \, \tilde{\theta}^2 \ge 0
\end{equation}

Derivando e utilizando a lei de adapta��o proposta, tem-se:
%
\begin{flalign}
\dot{V} &= \gamma^{-1} \, \tilde{\theta} \, \dot{\theta} \nonumber \\
&= - \frac{\varepsilon^2}{m^2} \leq 0 \,,
\end{flalign}
%
onde \eqref{eq:erro_est} foi utilizado. Como $\dot{V}$ � apenas negativa \textit{semi-definida}, mostra-se que $\tilde{\theta} \in \mathcal{L}_{\infty}$.

\begin{lemma}
Sejam $a,b \in \mathbb{R}$ e $m^2 = 1+a^2+b^2$. Ent�o:
%
\begin{flalign}
\norm{\frac{a}{m}} \leq 1 \,, \quad \forall a,b \in \mathbb{R} \,, \nonumber \\
\norm{\frac{b}{m}} \leq 1 \,, \quad \forall a,b \in \mathbb{R} \,. \nonumber
\end{flalign}
%
\end{lemma}

Dividindo \eqref{eq:erro_est} por $m = \sqrt{1+\zeta^2+\dot{\zeta}^2}$, obtemos:
%
\begin{flalign}
\frac{\varepsilon}{m} = \tilde{\theta} \, \frac{\zeta}{m} \,.
\label{eq:erro_est2}
\end{flalign}

De acordo com o Lema 1 (para $a = \zeta$, $b=\dot{\zeta}$), $\norm{\zeta/m} \leq 1$ e, como $\tilde{\theta} \in \mathcal{L}_{\infty}$, ent�o 
$\varepsilon/m \in \mathcal{L}_{\infty}$.
%
A lei de controle pode ser expressa por:
\begin{equation}
\dot{\theta} = \dot{\tilde{\theta}} = - \gamma \, \underbrace{\frac{\varepsilon}{m}}_{\in \mathcal{L}_{\infty}} \, \frac{\zeta}{m} \,.
\end{equation}

Como $\varepsilon/m \in \mathcal{L}_{\infty}$ e pelo Lema 1 $\norm{\zeta/m} \leq 1$, ent�o $\dot{\tilde{\theta}} \in \mathcal{L}_{\infty}$.
%
Finalmente, derivando \eqref{eq:erro_est} no tempo e divindo por $m$, obtemos:
%
\begin{flalign}
\frac{\dot{\varepsilon}}{m} = \underbrace{\dot{\tilde{\theta}}}_{\in \mathcal{L}_{\infty}} \, \frac{\zeta}{m} + \underbrace{\tilde{\theta}}_{\in \mathcal{L}_{\infty}} \, \frac{\dot{\zeta}}{m} \,.
\label{eq:erro_est3}
\end{flalign}

Novamente utilizando o Lema 1, mostra-se que os sinais constituintes de $\dot{\varepsilon}/m$ s�o limitados, e portanto $\dot{\varepsilon}/m \in \mathcal{L}_{\infty}$. $\blacksquare$
