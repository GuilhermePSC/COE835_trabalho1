%---------------------------------------------------------------------
\section{Resumo das equa��es do m�todo}

%Abaixo, resumimos algumas das principais equa��es utilizadas no m�todo.

\vspace{20mm}

\equacao{Planta}
  {\dot{y}_p = a_p y_p + u \label{eq:planta}}

\equacao{Modelo}
  {\dot{y}_m = -a_m y_m + r \label{eq:ref_model}}

\equacao{Erro de sa�da}
  {e_0 = y_p - y_m \label{eq:error}}

\equacao{Lei de controle}
  {u = \theta \, y_p + r \label{eq:ctrl_law}}

\equacao{Filtro}
  {\dot{\zeta} + a_m \, \zeta = y_p \label{eq:filter}}

\equacao{Estimativa do erro}
  {\hat{e} = \mathcal{L}^{-1}\left\{\frac{\theta y_p }{s+a_m}\right\}- \theta\zeta \label{eq:est_error}}

\equacao{Erro de estimativa}
  {\varepsilon = e_0 - \hat{e} \label{eq:error_est}}

\equacao{Sinal normalizante}
  {m^2 &= 1 + \zeta^2 \\ m^2 &= 1 + \zeta^2 + \dot{\zeta}^2 \label{eq:norm_signal}}

\equacao{Lei de adapta��o}
  {\dot{\theta} = - \frac{\gamma \varepsilon \zeta}{m^2} \label{eq:adpt_law}}

