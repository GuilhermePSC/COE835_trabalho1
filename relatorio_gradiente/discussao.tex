%---------------------------------------------------------------------
\section{Discuss�o}

O novo fator normalizante � rude para todos os casos.

O termo a mais no fator de normaliza��o vai sempre overnormalizar (limitar em exagero) e prejudicar a varia��o de $\theta$, ent�o converge mais lento. nao sei ainda qual a vantagem em ter $\dot{\epsilon}/m$ limitado.

No caso do af, da pra ver q aumentando af, vc diminui a varia��o de theta, ent�o o ajuste fica mais lento mesmo. Vc fica com uma equa��o (s+af)/(s2+2af+af2)